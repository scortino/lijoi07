\documentclass[11pt, handout]{beamer}
\usepackage[utf8]{inputenc}
\usepackage{datetime}
\usepackage{amsmath}
\usepackage{amssymb}
\usepackage{amsthm}
\usepackage{bbm}
\usepackage{bm}
\usepackage{hyperref}
\usepackage{environ}
\usepackage{mathtools}
\usepackage{tikz}

\newcommand{\circled}[1]{%
  \scalebox{0.6}{\tikz\node%
    [outer sep=0pt, inner sep=2pt,
      line width=0pt,text=white,fill=blue!50,draw,circle,shading=ball]{#1};%
  }%
}
\newcommand{\bref}[1]{\circled{\ref{#1}}}

\usetheme{CambridgeUS}
\usecolortheme{default}

\setbeamertemplate{theorems}[numbered]

\newcommand\Fontvi{\fontsize{10}{12}\selectfont}
\newcommand\Fontvii{\fontsize{9}{10}\selectfont}

\DeclareMathOperator*{\argmin}{argmin}

\newdate{date}{31}{01}{2022}

\title[Lijoi et al. (2007)]{Bayesian nonparametric estimation of the probability of discovering new species}
\subtitle{Lijoi, Mena \& Pr{\"u}nster (2007)}
\author{Stefano Cortinovis}
\date[20605 - Machine Learning II]{\displaydate{date}}

\AtBeginSection[]
{
    \begin{frame}
        \frametitle{Table of Contents}
        \tableofcontents[currentsection]
    \end{frame}
}

\begin{document}

\nocite{*}

\begin{frame}
\titlepage
\end{frame}

\section{Species sampling}

\begin{frame}[t]{Species sampling framework}
We formalize the process of drawing samples from a large population of individuals that can be grouped in different species as follows:
\begin{itemize}
    \item Let \((X_n)_{n \geq 1}\) be a sequence of random variables taking values in some set \(\mathbb{X}\).
    \begin{itemize}
        \item \(X_n\) represents the \textbf{species} of the \(n\)-th individual sampled
        \item \(\mathbb{X}\) represents an arbitrary set of \textbf{tags} used to label species
    \end{itemize}
    \item Define
    \begin{equation*}
        M_j \coloneqq 
        \begin{cases}
            1 & \text{if } j = 1\\
            \inf\{n \colon n > M_{j-1}, X_n \notin \{X_1,...,X_{n-1}\}\} & \text{if } j \geq 2
        \end{cases}
        % Mention inf{\emptyset} = inf and M_j < \infty
    \end{equation*}
    and, for \(M_j < \infty\), let \(\tilde{X}_j \coloneqq X_{M_j}\).
    \begin{itemize}
        \item \(\tilde{X}_j\) represents the \(j\)-th \textbf{distinct species} to be observed
    \end{itemize}
\end{itemize}
\end{frame}

\begin{frame}[t]{Species sampling framework}
    Formalize the process of drawing samples from a large population of individuals of various species as follows:
    \begin{itemize}
        \item<1-> Let \(K_n \coloneqq \max\{j \colon k \leq n \text{ and } M_j < \infty\}\).
        \begin{itemize}
            \item \(K_n\) represents the \textbf{number of distinct species} to appear in the first \(n\) observations
        \end{itemize}
        \item<2-> Define
        \begin{equation*}
            N_{jn} \coloneqq \sum_{i=1}^n \mathbbm{1}(X_i = \tilde{X}_j)
        \end{equation*}
        for \(j = 1,...,K_n\) and let \(\mathbf{N}_{n} = (N_{1n},...,N_{K_n n})\).
        \begin{itemize}
            \item \(N_{jn}\) represents the \textbf{number of times} that the \(j\)-th species \(\tilde{X}_j\) appears in the first \(n\) observations
        \end{itemize}
    \end{itemize}
\end{frame}

\begin{frame}{Species sampling problem}
    \Fontvi{}
    Taking inspiration from biological and ecological studies, given a sample of size \(n\) containing \(j\) distinct species, denoted by \(X_j^{(1,n)}\), we are interested in determining:
    \begin{enumerate}
        % Making inference about the number of unseen species
        \item\label{item:one} The probability distribution of the number of new species observed among the following \(m\) observations
        \begin{itemize}
            \item Denote the second unobserved sample of size \(m\) by \(X^{(2, m)} = (X_{n+1},...,X_{n+m})\)
            \item Denote the number of new species in \(X^{(2, m)}\) by \(K_m^{(n)} = K_{n + m} - K_n\)
            \item Then, \bref{item:one} amounts to determining \(\text{pr}(K_m^{(n)} = k | X_j^{(1,n)})\) for \(k = 0,...,m\) and for \(j = 1,...,n\) % sufficiency of K_n for gibbs type priors
        \end{itemize}
        % Estimating the probability that a further draw of m units from the population yields k new distinct species
        \item\label{item:two} The probability of observing a new species at the \((n + m + 1)\)-th draw
        \begin{itemize}
            \item Using the notation introduced above, \bref{item:two} amounts to determining the \textbf{random} probability \(D_m^{n:j} \coloneqq \text{pr}(K_1^{(m + n)} = 1 | X_j^{(1,n)}, X^{(2, m)})\) % randomness due to X^{(2, m)} 
        \end{itemize}
    \end{enumerate}
\end{frame}

\begin{frame}{Species Sampling Model (Pitman, 1996)}
    We call \((X_n)_{n \geq 1}\) a sample from random distribution \(\tilde{P}\) if \(X_1,X_2,... | \tilde{P} \overset{iid}{\sim} \tilde{P}\).
    \begin{definition}[Species sampling model]
        \label{th:ssm}
        Let \((X_n)_{n \geq 1}\) be a sample from a discrete random distribution \(\tilde{P}\) of the form
        \begin{equation}
            \label{eq:ssp}
            \tilde{P} = \sum_{i = 1}^{\infty} P_{i} \delta_{\hat{X}_i}
        \end{equation}
        where \((P_i)_{i \geq 1}\) is a sequence of random variables such that \(P_i \geq 0\) a.s. for every \(i\), \(\sum_i P_i = 1 \text{ a.s.}\), and \((\hat{X}_i)_{n \geq 1} \overset{iid}{\sim} \nu\) independently of \((P_i)\) for \(\nu\) diffuse.
    \end{definition}
    \medskip
    \pause
    % Here P_i should be interpreted as the frequency of the i-th species in some listing of the species in the population and \hat{X}_i is the tag associated with that species.
    The setup above, with \(\tilde{P}\) as in \eqref{eq:ssp} and \((X_n)_{n \geq 1} | \tilde{P} \overset{iid}{\sim} \tilde{P}\) is called a (proper) \textit{species sampling model} (\textbf{SSM}) with \textit{species sampling process} (\textbf{SSP}) \(\tilde{P}\).
    % Given that \nu is diffuse, the labels \hat{X_i} are all different almost surely. Therefore, F has atoms P_i as the random measure in Kingman's representation theorem. Hence, the partitions generated by a species sampling model can be thought of as a general model for IERP.
    % That is to say, F is discrete almost surely
\end{frame}

\begin{frame}{SSM Characterization}
    As a result of the discrete nature of \(\tilde{P}\), an SSM induces an infinite exchangeable random partition.
    \medskip
    
    The strong link between a species sampling model and the partition it induces was unveiled by Pitman, who showed that the former is characterized by \(\nu\) and by the exchangeable partition probability function (EPPF) associated with the latter. 
    \medskip
    
    In other words, for a given \(\nu\), an SSM is characterized by the joint distributions of \(K_n\) and \(\mathbf{N}_n\), i.e.
    \begin{equation*}
        \text{pr}[\{K_n = k\} \cap \{N_{jn} = n_j,\ j = 1,...,k\}],
    \end{equation*}
    for \(n \geq 1\).
\end{frame}

\section{Gibbs-type priors}

\section{Estimating the probability of discovering a new species}

\section{Numerical example}

\section*{}
\begin{frame}[allowframebreaks] %allow to expand references to multiple frames (slides)
\frametitle{References}
\scriptsize{\bibliographystyle{acm}}
\bibliography{lijoi07}
\end{frame}
 
\end{document}