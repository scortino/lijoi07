\documentclass[11pt, handout]{beamer}
\usepackage[utf8]{inputenc}
\usepackage{datetime}
\usepackage{amsmath}
\usepackage{amssymb}
\usepackage{amsthm}
\usepackage{bbm}
\usepackage{bm}
\usepackage{hyperref}
\usepackage{environ}
\usepackage{mathtools}

\usetheme{CambridgeUS}
\usecolortheme{default}

\setbeamertemplate{theorems}[numbered]

\newcommand\Fontvi{\fontsize{10}{12}\selectfont}
\newcommand\Fontvii{\fontsize{9}{10}\selectfont}

\DeclareMathOperator*{\argmin}{argmin}

\newdate{date}{31}{01}{2022}

\title[Lijoi et al. (2007)]{Bayesian nonparametric estimation of the probability of discovering new species}
\subtitle{Lijoi, Mena \& Pr{\"u}nster (2007)}
\author{Stefano Cortinovis}
\date[20605 - Machine Learning II]{\displaydate{date}}

\AtBeginSection[]
{
    \begin{frame}
        \frametitle{Table of Contents}
        \tableofcontents[currentsection]
    \end{frame}
}

\begin{document}

\nocite{*}

\begin{frame}
\titlepage
\end{frame}

\section{Species sampling}

\begin{frame}[t]{Species sampling framework}
We formalize the process of drawing samples from a large population of individuals that can be grouped in different species as follows:
\begin{itemize}
    \item Let \((X_n)_{n \geq 1}\) be a sequence of random variables taking values in some set \(\mathbb{X}\).
    \begin{itemize}
        \item \(X_n\) represents the species of the \(n\)-th individual sampled
        \item \(\mathbb{X}\) represents an arbitrary set of tags used to label species
    \end{itemize}z
    \item Define
    \begin{equation*}
        M_j \coloneqq 
        \begin{cases}
            1 & \text{if } j = 1\\
            \inf\{n \colon n > M_{j-1}, X_n \notin \{X_1,...,X_{n-1}\}\} & \text{if } j \geq 2
        \end{cases}
        % Mention inf{\emptyset} = inf and M_j < \infty
    \end{equation*}
    and, for \(M_j < \infty\), let \(\tilde{X}_j \coloneqq X_{M_j}\).
    \begin{itemize}
        \item \(\tilde{X}_j\) represents the \(j\)-th \textbf{distinct species} to be observed
    \end{itemize}
\end{itemize}
\end{frame}

\begin{frame}[t]{Species Sampling}
    Formalize the process of drawing samples from a large population of individuals of various species as follows:
    \begin{itemize}
        \item<1-> Let \(K_n \coloneqq \max\{j \colon k \leq n \text{ and } M_j < \infty\}\).
        \begin{itemize}
            \item \(K_n\) represents the \textbf{number of distinct species} to appear in the first \(n\) observations
        \end{itemize}
        \item<2-> Define
        \begin{equation*}
            N_{jn} \coloneqq \sum_{i=1}^n \mathbbm{1}(X_i = \tilde{X}_j)
        \end{equation*}
        for \(j = 1,...,K_n\) and let \(\mathbf{N}_{n} = (N_{1n},...,N_{K_n n})\).
        \begin{itemize}
            \item \(N_{jn}\) represents the \textbf{number of times} that the \(j\)-th species \(\tilde{X}_j\) appears in the first \(n\) observations
        \end{itemize}
    \end{itemize}
\end{frame}

\section{Gibbs-type priors}

\section{Estimating the probability of discovering a new species}

\section{Numerical example}

\section*{}
\begin{frame}[allowframebreaks] %allow to expand references to multiple frames (slides)
\frametitle{References}
\scriptsize{\bibliographystyle{acm}}
\bibliography{lijoi07}
\end{frame}
 
\end{document}